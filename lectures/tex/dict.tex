\section{Dictionaries} % (fold)
\label{sec:dictionaries}

\begin{frame}\frametitle{Dictionaries}
    \framesubtitle{}

    A dictionary is a \emph{collection} of \emph{key-value} pairs.

    \vfill

    An example: the keys are all words in the English language, and their
    corresponding values are the meanings.

    \vfill

    Lists + Dictionaries = \$\$\$

\end{frame}

\begin{frame}\frametitle{Defining a dictionary}
    \framesubtitle{}

    \codeblock{code/dict_def.py}

    Note how we can add more key-value pairs at any time.
    Also, only condition on keys is that they are \emph{immutable}.

\end{frame}

\begin{frame}\frametitle{No duplicate keys}
    \framesubtitle{}

    Old value gets overwritten instead!

    \codeblock{code/dict_duplicate_keys.py}

\end{frame}

\begin{frame}\frametitle{Access}
    \framesubtitle{}

    We can access values by keys, but not the other way around

    \codeblock{code/dict_access.py}

    \pause

    Furthermore, we can check whether a key is in the dictionary by\\
    \texttt{key in dict}

\end{frame}

% \begin{frame}\frametitle{Deleting elements}
%     \framesubtitle{}

%     We can remove a key-value pair by key using \texttt{del}. And we can \texttt{clear}
%     the dictionary.

%     \codeblock{code/dict_del.py}

% \end{frame}

\begin{frame}\frametitle{All keys, values or both}
    \framesubtitle{}

    Use \texttt{d.keys()}, \texttt{d.values()} and \texttt{d.items()}

    \codeblock{code/dict_loops.py}

    So how can you loop over dictionaries?

\end{frame}

\begin{frame}\frametitle{Small exercise}
    \framesubtitle{}

    Print all key-value pairs of a dictionary

    \pause

    \codeblock{code/dict_print.py}

    Instead of \texttt{d.items()}, you can use \texttt{d.iteritems()} as well.
    Better performance for large dictionaries.

\end{frame}

% section dictionaries (end)
