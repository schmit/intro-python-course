
\section{Administrivia} % (fold)
\label{sec:administrivia}

% \begin{frame}\frametitle{Instructor}
%     \framesubtitle{}

%     Sven Schmit

%     \begin{itemize}
%         \item from the Netherlands
%         \item Third year PhD student in ICME
%         \item Background in Statistics / Mathematics / CS
%         \item Used Python for data science at Stitch Fix
%     \end{itemize}

% \end{frame}

\begin{frame}\frametitle{Quick poll}

    Who has...
    \begin{itemize}
        \item written one line of code?
        \pause
        \item written a \texttt{for} loop?
        \pause
        \item written a \texttt{function}?
        \pause
        \item heard of recursion?
        \pause
        \item heard of object oriented programming?
        \pause
        \item unit testing?
    \end{itemize}
% Know nothing: this might be too difficult
% Know all: might be too easy
\end{frame}

\begin{frame}\frametitle{Feedback}

    If you have comments, like things to be done differently, please let me know
    and let me know asap.

    \vfill

    Questionnaires at the end of the quarter are nice, but they won't help you.

\end{frame}

\begin{frame}\frametitle{Content of course}
\begin{itemize}
    \item Variables
    \item Functions
    \item Data types
    \begin{itemize}
        \item Strings, Lists, Tuples, Dictionaries
    \end{itemize}
    \item File input and output (I/O)
    \item Classes
    \item Exception handling
    \item Recursion
    \item Numpy, Scipy and Matplotlib
    \item Pandas, Statsmodels and IPython
    \item Unit tests
    \item More packages
\end{itemize}

\end{frame}

\begin{frame}\frametitle{Setup of course}

\begin{itemize}
    \item Lectures: first half lecture, second half exercises
    \item Portfolio
    \item Final project
\end{itemize}

\end{frame}

\begin{frame}\frametitle{More abstract setup of course}

    My job is to show you the possibilities and some resources (exercises etc.)

    \vfill

    Your job is to teach yourself Python

    \vfill

    If you think you can learn Python by just listening to me,
    you are grossly overestimating my abilities.

\end{frame}

\begin{frame}\frametitle{Exercises}
    \framesubtitle{}

    Exercises in second half of the class. Try to finish them in class,
    else make sure you understand them all before next class.

    \vfill

    At the end of the course, hand in a portfolio with your solutions for exercises.

    \vfill

    Feel free (or: You are strongly encouraged) to work in pairs on the exercises.
    It's acceptable to hand in the same copy of code for your portfolio if you work in pairs,
    but do mention your partner.

\end{frame}

\begin{frame}\frametitle{Portfolio}
    \framesubtitle{}

    You are required to hand in a portfolio with your solutions to the exercises you attempted.

    \vfill

    This is to show your active participation in class

    \vfill

    You are expected to do at least 2/3rd of the assigned exercises.
    Feel free to skip some problems, for example if you lack some required math background knowledge.

    \vfill

    Don't worry about this now, just save all the code you write.

\end{frame}

\begin{frame}\frametitle{Final project}

    Besides the portfolio, you are required to submit a project,
    due one week after the final class.

    \vfill

    One paragraph proposal due before lecture 4.

    \vfill

    Have fun by working on a project that interests you while learning Python.

    \vfill

    You are encouraged to use material not taught in class:
    we can't cover everything in class.

    \vfill

    No teams :(


\end{frame}

\begin{frame}\frametitle{Final project ideas}

    Some more pointers

    \begin{itemize}
        \item See course website for some possible projects
        \item Focus on Python, not the application (no research)
    \end{itemize}

    Some successful previous projects

    \begin{itemize}
        \item Predicting stock movement, obtaining data using Quandl Api
        \item Crawling and analyzing data from a large forum
        \item Finding recent earthquakes close to a given location using Google maps API and USGS data
    \end{itemize}


\end{frame}

\begin{frame}\frametitle{Workload}

    The only way to learn Python, is by writing Python... \textbf{a lot}.
    So you are expected to put in effort.

    \vfill

    From past experience: If you are new to programming, consider this a hard 3 unit class
    where you will have to figure out quite a bit on your own.
    However, if you have a solid background in another language, this class should be pretty easy.

\end{frame}

\begin{frame}\frametitle{More on workload}

    Expect the first 4 weeks to be very intense: cover all basics.

    After that: reap rewards, focus on project, not as much exercises.

    \vfill

    So: hang on tight, it'll be worth it.

\end{frame}

\begin{frame}\frametitle{To new programmers}

    Be warned, this will be difficult.

    \vfill

    The problem: 8 weeks, 8 lectures, 1 unit. We will simply go too fast.

    \vfill

    Alternative: spend some time learning on your own (Codecademy / Udacity etc).
    There are so many excellent resources online these days.

    Then you can always come back in Fall; this course will be offered again.

\end{frame}

\begin{frame}\frametitle{Misc}

\begin{description}
    \item[Website] \url{stanford.edu/\~schmit/cme193}
    \item[Piazza] Use Piazza for discussing problems. An active user on Piazza has more leeway when it comes to portfolio, as it shows involvement.
    \item[Office hours] After class or by appointment (shoot me an email).
\end{description}

\end{frame}

\begin{frame}\frametitle{References}
    \framesubtitle{}

    The internet is an excellent source, and Google is a perfect starting point.

    \vfill

    The official documentation is also good, always worth a try:
    \url{https://docs.python.org/2/}.

    \vfill

    Course website has a list of useful references.

\end{frame}

\begin{frame}\frametitle{Last words before we get to it}
    \framesubtitle{}

    \begin{itemize}
        \item Work
        \item Make friends
        \item Fail often
        \item Fail gently
    \end{itemize}

\end{frame}

% section administrivia (end)