\section{File I/O} % (fold)
\label{sec:file_i_o}

\begin{frame}\frametitle{File I/O}
    \framesubtitle{}

    How to read from and write to disk.

\end{frame}

\begin{frame}\frametitle{The file object}
    \framesubtitle{}

    \begin{itemize}
        \item Interaction with the file system is pretty straightforward in Python.
        \item Done using \emph{file objects}
        \item We can instantiate a file object using \texttt{open} or \texttt{file}
    \end{itemize}

\end{frame}

\begin{frame}\frametitle{Opening a file}
    \framesubtitle{}

    \texttt{f = open(filename, option)}

    \vfill

    \begin{itemize}
        \item filename: path and filename
        \item option:
            \begin{description}
                \item['r'] read file
                \item['w'] write to file
                \item['a'] append to file
            \end{description}
    \end{itemize}

    \vfill

    We need to close a file after we are done:
    \texttt{f.close()}

\end{frame}

\begin{frame}\frametitle{with open() as f}
    \framesubtitle{}

    Very useful way to open, read/write and close file:
    \codeblock{code/fileio_with.py}

\end{frame}

\begin{frame}\frametitle{Reading files}
    \framesubtitle{}

    \begin{description}
        \item[read()] Read entire line (or first $n$ characters, if supplied)
        \item[readline()] Reads a single line per call
        \item[readlines()] Returns a list with lines (splits at newline)
    \end{description}

    \pause

    Another fast option to read a file
    \codeblock{code/fileio_loopread.py}

\end{frame}

\begin{frame}\frametitle{Writing to file}
    \framesubtitle{}

    Use \texttt{write()} to write to a file

    \codeblock{code/fileio_write.py}

\end{frame}

\begin{frame}\frametitle{More writing examples}
    \framesubtitle{}

    \codeblock{code/fileio_writelist.py}

\end{frame}

% section file_i_o (end)
