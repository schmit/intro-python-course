\section{Basics} % (fold)
\label{sec:basics}

\begin{questions}
\titledquestion{The interpreter}
Open the Python interpeter.
What happens when you input the following statements:
    \begin{parts}
        \part \texttt{3 + 1}
        \part \texttt{3 * 3}
        \part \texttt{2 ** 3}
        \part \texttt{"Hello, world!"}
    \end{parts}

\titledquestion{Scripts}
    Now copy the above to a script, and save it as \texttt{script1.py}.
    What happens if you run the script? (try: \texttt{python script1.py}).
    Can you fix this (hint: use the \texttt{print} function)

\titledquestion{More interpreter}
    Explain the output of the following statements if executed subsequently:
    \begin{parts}
        \part \texttt{'py' + 'thon'}
        \part \texttt{'py' * 3 + 'thon'}
        \part \texttt{'py' - 'py'}
        \part \texttt{'3' + 3}
        \part \texttt{3 * '3'}
        \part \texttt{a}
        \part \texttt{a = 3}
        \part \texttt{a}
    \end{parts}

\titledquestion{Booleans}
    Explain the output of the following statements:
    \begin{parts}
        \part \texttt{1 == 1}
        \part \texttt{1 == True}
        \part \texttt{0 == True}
        \part \texttt{0 == False}
        \part \texttt{3 == 1 * 3}
        \part \texttt{(3 == 1) * 3}
        \part \texttt{(3 == 3) * 4 + 3 == 1}
        \part \texttt{3**5 $>=$ 4**4}
    \end{parts}


\titledquestion{Integers}
    Explain the output of the following statements:
    \begin{parts}
        \part \texttt{5 / 3}
        \part \texttt{5 \% 3}
        \part \texttt{5.0 / 3}
        \part \texttt{5 / 3.0}
        \part \texttt{5.2 \% 3}
        \part \texttt{2001 ** 200}
    \end{parts}

\titledquestion{Floats}
    Explain the output of the following statements:
    \begin{parts}
        \part \texttt{2000.3 ** 200} (compare with above)
        \part \texttt{1.0 + 1.0 - 1.0}
        \part \texttt{1.0 + 1.0e20 - 1.0e20}
    \end{parts}

\titledquestion{Variables}
    Write a script where the variable \texttt{name} holds a string with your name.
    Then, assuming for now your name is \emph{John Doe}, have the script output:
    \texttt{Hello, John Doe!} (and obviously, do not use
    \texttt{print "Hello, John Doe!"}.

\titledquestion{Type casting}

    Very often, one wants to ``cast'' variables of a certain type into another type.
    Suppose we have variable \texttt{x = '123'}, but really we would like \texttt{x} to be
    an integer.

    This is easy to do in Python, just use \texttt{desiredtype(x)}, e.g. \texttt{int(x)}
    to obtain an integer.

    Try the following and explain the output
    \begin{parts}
        \part \texttt{float(123)}
        \part \texttt{float('123')}
        \part \texttt{float('123.23')}
        \part \texttt{int(123.23)}
        \part \texttt{int('123.23')}
        \part \texttt{int(float('123.23'))}
        \part \texttt{str(12)}
        \part \texttt{str(12.2)}
        \part \texttt{bool('a')}
        \part \texttt{bool(0)}
        \part \texttt{bool(0.1)}
    \end{parts}

\end{questions}
